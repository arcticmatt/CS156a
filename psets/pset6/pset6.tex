%%%%%%%%%%%%%%%%%%%%%%%%%%%%%% Preamble
\documentclass{article}
\usepackage{amsmath,amssymb,amsthm,fullpage}
\usepackage[a4paper,bindingoffset=0in,left=1in,right=1in,top=1in,
bottom=1in,footskip=0in]{geometry}
\newtheorem*{prop}{Proposition}
%\newcounter{Examplecount}
%\setcounter{Examplecount}{0}
\newenvironment{discussion}{\noindent Discussion.}{}
\setlength{\headheight}{12pt}
\setlength{\headsep}{10pt}
\usepackage{fancyhdr}

\pagestyle{fancy}
\fancyhf{}
\lhead{CS156a Pset 6}
\rhead{Matt Lim}
\begin{document}

%%%%%%%%%%%%%%%%%%%%%%%%%%%%%% Problem 1
\section*{Problem 1}
\textbf{b} is the correct answer.

\noindent This is because when we use a less sophistcated hypothesis set,
we are going to be able, in general, to capture less of the target function
$f$.

%%%%%%%%%%%%%%%%%%%%%%%%%%%%%% Problem 2
\section*{Problem 2}
\textbf{a} is the correct answer.

\noindent I wrote some code to do it. More specifically, I wrote code
that runs Linear Regression on the training set after performing the non-linear
transformation. This basically just boils down to calculating the weight vector
by multiplying a bunch of matrices, as seen in lecture 12 slide 11.
I then calculated the in-sample and out-of-sample classification
errors in the usual manner (using the weight vector I calculated, calculating
dot products, and comparing signs).

%%%%%%%%%%%%%%%%%%%%%%%%%%%%%% Problem 3
\section*{Problem 3}
\textbf{d} is the correct answer.

\noindent I wrote some code to do it. More specifically, I wrote code that
adds weight decay to Linear Regression. This basically just boils down to
calculating the weight vector $w_{reg}$ by multiplying a bunch of matrices,
as seen in lecture 12 slide 11. In this multiplication, I used the term
given to us in the problem and $k=-3$. I then calculated the in-sample and
out-of-sample classification errors in the usual manner.

%%%%%%%%%%%%%%%%%%%%%%%%%%%%%% Problem 4
\section*{Problem 4}
\textbf{e} is the correct answer.

\noindent I wrote some code to do it. More specifically, I wrote code that
adds weight decay to Linear Regression. This basically just boils down to
calculating the weight vector $w_{reg}$ by multiplying a bunch of matrices,
as seen in lecture 12 slide 11. In this multiplication, I used the term
given to us in the problem and $k=3$. I then calculated the in-sample and
out-of-sample classification errors in the usual manner.

%%%%%%%%%%%%%%%%%%%%%%%%%%%%%% Problem 5
\section*{Problem 5}
\textbf{d} is the correct answer.

\noindent I wrote some code to do it. More specifically, I used the code
I wrote for the previous two parts and calculated the out-of-sample
classification error for each of the $k$ values given in the choices. I got
the smallest out-of-sample classification error when $k = -1$.

%%%%%%%%%%%%%%%%%%%%%%%%%%%%%% Problem 6
\section*{Problem 6}
\textbf{b} is the correct answer.

\noindent I wrote some code to do it. More specifically, I looped through
a large range of $k$s (positive and negative) and kept track of the minimum
out-of-sample classification error. The minimum I found was $.056$, which is
closest to $.06$.

%%%%%%%%%%%%%%%%%%%%%%%%%%%%%% Problem 7
\section*{Problem 7}
\textbf{c} is the correct answer.

\noindent \textbf{c} is the correct answer because this intersection makes
it so that when $q > 2$, $w_q = 0$ (for lower terms, the weights are free to
be anything). This is because when we take the intersection
we just get the set on the left because it is included in the set on the right.
This is clear to see because the set on the left forces all terms where
$q \geq 3$ to be zero, and the set on the right forces all terms where
$q \geq 4$ to be zero. So the set on the left is clearly the intersection of
both the sets, and since the set on the left is exactly what $\mathcal{H}_2$ is,
\textbf{c} is our answer.

%%%%%%%%%%%%%%%%%%%%%%%%%%%%%% Problem 8
\section*{Problem 8}
\textbf{e} is the correct answer.

\noindent For the operations including $w_{ij}^{(l)}x_i^{(l-1)}$, which
are of the form
\[ x_j^{(l)} = \theta(\sum_{i=0}^{d^{(l-1)}} w_{ij}^{(l)} x_i^{(l-1)}) \]
I calculated $22$ operations ($18$ for $l = 1$, $1 \leq j \leq 3$ and $4$
for $l=2$, $1 \leq j \leq 1$). For the operations including
$w_{ij}^{(l)}\delta_{j}^{(l)}$, where are of the form
\[ d_i^{(l-1)} = (1-(x_i^{(l-1)})^2) \sum_{j=1}^{d^{(l)}} w_{ij}^{(l)} d_j^{(l)} \]
I calculated $18$ operations ($15$ for $l=1$,
$1 \leq i \leq 5$ and $3$ for $l=2$, $1 \leq i \leq 3$; $i=0$ was not counted
because here the coefficient
in front of the sum is just zero). For the operations including $x_i^{(l-1)}
d_j^{(l)}$, I calculated more than $10$ operations, which puts my answer
choice at \textbf{e}.

%%%%%%%%%%%%%%%%%%%%%%%%%%%%%% Problem 9
\section*{Problem 9}
\textbf{a} is the correct answer.

\noindent Since each layer is fully connected, we only want to put two units
on each hidden layer. Since no weights go into the $x_0^{(l)}$s, we have that
this gives us $10$ weights from the input units to the first hidden layer and
$36$ weights total from the hidden units to other hidden units/the output unit
(one weight per hidden unit, as for each hidden layer, we will have one weight
going from each hidden unit to the non $x_0^{(l)}$ unit of the next layer, or
in the case of the last hidden layer, going to the output unit).

%%%%%%%%%%%%%%%%%%%%%%%%%%%%%% Problem 10
\section*{Problem 10}
\textbf{e} is the correct answer.

\noindent It seemed like 2 hidden levels would maximize the number of weights.
So I came up with a function for this situation,
\[ |w| = 10(x-1) + x(36-x-1) + (36-x) \]
where $x$ is the number of units in the first hidden level and $|w|$ is the
number of weights in the network. This function is based off of the fact that each
layer is fully connected to the next layer. I then maximized this function in
Wolfram Alpha and it gave me that $|w|_{max} = 510$ when $x=22$. And since no
answers are greater than $510$, this must be our answer.
\end{document}
