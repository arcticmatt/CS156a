%%%%%%%%%%%%%%%%%%%%%%%%%%%%%% Preamble
\documentclass{article}
\usepackage{amsmath,amssymb,amsthm,fullpage}
\usepackage[a4paper,bindingoffset=0in,left=1in,right=1in,top=1in,
bottom=1in,footskip=0in]{geometry}
\newtheorem*{prop}{Proposition}
%\newcounter{Examplecount}
%\setcounter{Examplecount}{0}
\newenvironment{discussion}{\noindent Discussion.}{}
\setlength{\headheight}{12pt}
\setlength{\headsep}{10pt}
\usepackage{fancyhdr}

\pagestyle{fancy}
\fancyhf{}
\lhead{CS156a Pset 2}
\rhead{Matt Lim}
\begin{document}
%%%%%%%%%%%%%%%%%%%%%%%%%%%%%% Problem 1
\section*{Problem 1}
\textbf{b} is the correct answer.

\noindent I got this answer by running the experiment detailed in the problem 100,000
times, and averaging the $\nu_{min}$.
%%%%%%%%%%%%%%%%%%%%%%%%%%%%%% Problem 2
\section*{Problem 2}
\textbf{d} is the correct answer.

\noindent In the simulation I ran, my averages for $\nu_1$ and $\nu_{rand}$ were
basically $.5$. And since $\mu = .5$, we have that the coins $c_1$ and $c_{rand}$
satisfy the Hoeffding Inequality.
%%%%%%%%%%%%%%%%%%%%%%%%%%%%%% Problem 3
\section*{Problem 3}
\textbf{e} is the correct answer.

\noindent The first way to get an error is when $y = f(x)$ but
the hypothesis makes an error in approximating the deterministic function.
The probability of this happening is $\lambda * \mu$. The second way to get
an error is when $y \neq f(x)$ and the hypothesis does not
make an error in approximating the deterministic function. The probability of
this happening is $(1 - \lambda) * (1 - \mu)$. So overall we get that the
probability of error that $h$ makes in approximating $y$ is $(1 - \lambda) *
(1 - \mu) + \lambda * \mu$.
%%%%%%%%%%%%%%%%%%%%%%%%%%%%%% Problem 4
\section*{Problem 4}
\textbf{e} is the correct answer.

\noindent By looking at our answer to Problem 3, which is $2*\lambda*\mu - \lambda
- \mu + 1$ which is equivalent to $(1 - \lambda) * (1 - \mu) + \lambda * \mu$,
, we can see that at no values of $\lambda$ will the performance of $h$
be independent of $\mu$ (we can't get rid of $\mu$ in that equation by setting
$\lambda$ to any of the chooseable values).
%%%%%%%%%%%%%%%%%%%%%%%%%%%%%% Problem 5
\section*{Problem 5}
\textbf{c} is the correct answer.

\noindent I used Linear Regression to find the weight vector for some 100
points as described in the problem, and used this
weight vector to calculate the fraction of misclassified in-sample points. I
then repeated this 1000 times and took the average, and \textbf{c} was the closest
number to the average I got.
%%%%%%%%%%%%%%%%%%%%%%%%%%%%%% Problem 6
\section*{Problem 6}
\textbf{c} is the correct answer.

\noindent I used Linear Regression to find the weight vector for some 100 points
as described in the problem. I then generated 1000 fresh points, and used the
weight vector to see what fraction of these points were misclassified. I then
repeated this 1000 times and took the average, and \textbf{c} was the closest number
to the average I got.
%%%%%%%%%%%%%%%%%%%%%%%%%%%%%% Problem 7
\section*{Problem 7}
\textbf{a} is the correct answer.

\noindent I used Linear Regression to find the weight vector for some 10 points
as described in the problem. I then ran the PLA on those points, using the
weight vector I found as the initial weight vector in this algorithm.
I then repeated this 1000 times and took the average number of iterations of the
PLA, and \textbf{a} was the closest number to the average I got.
%%%%%%%%%%%%%%%%%%%%%%%%%%%%%% Problem 8
\section*{Problem 8}
\textbf{d} is the correct answer.

\noindent I generated a training set of 1000 points, assigned them scores according
to the target function given in the problem, and generated simulated noise as
instructed in the problem. I then used Linear Regression to find the weight
vector for this set of points, and used the weight vector to calculate the
fraction of misclassified in-sample points. I then repeated this 1000 times and took
the average, and \textbf{d} was the closest to the average I got.
%%%%%%%%%%%%%%%%%%%%%%%%%%%%%% Problem 9
\section*{Problem 9}
\textbf{a} is the correct answer.

\noindent I transformed the training data into nonlinear feature vectors as
described in the problem. I then used Linear Regression on the set of
feature vectors to find the weight vector. I repeated this 1000 times, and
took the average weight vector. \textbf{a} was the closest to the average
weight vector I got. I also checked this by comparing the \textbf{a-e} hypotheses
point by point (with a batch of test points) to the hypothesis I found, and
\textbf{a} disagreed on the least amount of points.
%%%%%%%%%%%%%%%%%%%%%%%%%%%%%% Problem 10
\section*{Problem 10}
\textbf{b} is the correct answer.

\noindent I did the same as in Problem 9, running Linear Regression on the
set of feature vectors to find the weight vector. Then, I generated a new
set of 1000 points and added noise as before. By using the output of these points,
the feature vectors of these points, and the weight vector, I was able to find the
fraction of the points that were misclassified. I then repeated this 1000 times and
took the average, and \textbf{b} was the closest to the average I got.
\end{document}
