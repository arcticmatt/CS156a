%%%%%%%%%%%%%%%%%%%%%%%%%%%%%% Preamble
\documentclass{article}
\usepackage{amsmath,amssymb,amsthm,fullpage}
\usepackage[a4paper,bindingoffset=0in,left=1in,right=1in,top=1in,
bottom=1in,footskip=0in]{geometry}
\newtheorem*{prop}{Proposition}
%\newcounter{Examplecount}
%\setcounter{Examplecount}{0}
\newenvironment{discussion}{\noindent Discussion.}{}
\setlength{\headheight}{12pt}
\setlength{\headsep}{10pt}
\usepackage{fancyhdr}

\pagestyle{fancy}
\fancyhf{}
\lhead{CS156a Pset 5}
\rhead{Matt Lim}
\begin{document}

%%%%%%%%%%%%%%%%%%%%%%%%%%%%%% Problem 1
\section*{Problem 1}
\textbf{c} is the correct answer.

\noindent When $N = 25$, the expected $E_{in}$ is $.0064$. When $N = 100$,
the expected $E_{in}$ is $.0091$. Thus we have our answer.

%%%%%%%%%%%%%%%%%%%%%%%%%%%%%% Problem 2
\section*{Problem 2}
\textbf{d} is the correct answer.

\noindent We are looking to find the correct behavior for
$\text{sign}(\tilde{w}_0 + \tilde{w}_1x_1^2 + \tilde{w}_2x_2^2)$. We have
that if $\tilde{w}_1 < 0$ and $\tilde{w}_2 > 0$, then we get a positive
sign for large absolute values of $x_2$ and a negative sign for large absolute
values of $x_1$. And we can adjust $\tilde{w}_0$ to achieve the exact
behavior we are looking for. Thus we have our answer.

%%%%%%%%%%%%%%%%%%%%%%%%%%%%%% Problem 3
\section*{Problem 3}
\textbf{c} is the correct answer.

\noindent We have that the VC dimension of a linear model in the transformed
space $d_{VC} \leq \tilde{d} + 1$. And we have that $\tilde{d} = 14$, which
means that $d_{VC} \leq 15$. Thus we have our answer.

%%%%%%%%%%%%%%%%%%%%%%%%%%%%%% Problem 4
\section*{Problem 4}
\textbf{e}

\noindent We just use the chain rule to take the partial derivative of
$E(u,v)$ with respect to $u$ to get the answer.

%%%%%%%%%%%%%%%%%%%%%%%%%%%%%% Problem 5
\section*{Problem 5}

%%%%%%%%%%%%%%%%%%%%%%%%%%%%%% Problem 6
\section*{Problem 6}

%%%%%%%%%%%%%%%%%%%%%%%%%%%%%% Problem 7
\section*{Problem 7}

%%%%%%%%%%%%%%%%%%%%%%%%%%%%%% Problem 8
\section*{Problem 8}

%%%%%%%%%%%%%%%%%%%%%%%%%%%%%% Problem 9
\section*{Problem 9}

%%%%%%%%%%%%%%%%%%%%%%%%%%%%%% Problem 10
\section*{Problem 10}
\end{document}
