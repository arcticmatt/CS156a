%%%%%%%%%%%%%%%%%%%%%%%%%%%%%% Preamble
\documentclass{article}
\usepackage{amsmath,amssymb,amsthm,fullpage}
\usepackage[a4paper,bindingoffset=0in,left=1in,right=1in,top=1in,
bottom=1in,footskip=0in]{geometry}
\newtheorem*{prop}{Proposition}
%\newcounter{Examplecount}
%\setcounter{Examplecount}{0}
\newenvironment{discussion}{\noindent Discussion.}{}
\setlength{\headheight}{12pt}
\setlength{\headsep}{10pt}
\usepackage{fancyhdr}

\pagestyle{fancy}
\fancyhf{}
\lhead{CS156a Pset 3}
\rhead{Matt Lim}
\begin{document}

%%%%%%%%%%%%%%%%%%%%%%%%%%%%%% Problem 1
\section*{Problem 1}
\textbf{b} is the correct answer.

\noindent We want the probability bound $2Me^{2\epsilon^2N}$ to be at most $.03$.
So, we can solve for
\[ 2Me^{-2\epsilon^2N} = .03 \]
\[ 2e^{-2(.05)^2N} = .03 \]
\[ e^{-2(.05)^2N} = .015 \]
\[ N = 839.94 \]
So for any $N$ less than this value, our probability bound will be greater
than $.03$. So the answer is $1000$. We can check this by plugging in
$500$ and $1000$ into $2Me^{2\epsilon^2N}$. At $500$, we get $.164$, which is
too big. At $1000$, we get $.013$, which is below the desired bound. Thus
$1000$ is the correct answer.

%%%%%%%%%%%%%%%%%%%%%%%%%%%%%% Problem 2
\section*{Problem 2}
\textbf{c} is the correct answer.

\noindent We want the probability bound $2Me^{2\epsilon^2N}$ to be at most $.03$.
So, we can solve for
\[ 2Me^{-2\epsilon^2N} = .03 \]
\[ 20e^{-2(.05)^2N} = .03 \]
\[ e^{-2(.05)^2N} = .0015 \]
\[ N = 1300.46 \]
So for any $N$ less than this value, our probability bound will be greater
than $.03$. So the answer is $1500$. We can check this by plugging in
$1000$ and $1500$ into $2Me^{2\epsilon^2N}$. At $1000$, we get $.135$, which is
too big. At $1500$, we get $.011$, which is below the desired bound. Thus
$1500$ is the correct answer.

%%%%%%%%%%%%%%%%%%%%%%%%%%%%%% Problem 3
\section*{Problem 3}
\textbf{d} is the correct answer.

\noindent We want the probability bound $2Me^{2\epsilon^2N}$ to be at most $.03$.
So, we can solve for
\[ 2Me^{-2\epsilon^2N} = .03 \]
\[ 200e^{-2(.05)^2N} = .03 \]
\[ e^{-2(.05)^2N} = .00015 \]
\[ N = 1760.98 \]
So for any $N$ less than this value, our probability bound will be greater
than $.03$. So the answer is $2000$. We can check this by plugging in
$1500$ and $2000$ into $2Me^{2\epsilon^2N}$. At $1500$, we get $.111$, which is
too big. At $2000$, we get $.009$, which is below the desired bound. Thus
$2000$ is the correct answer.

%%%%%%%%%%%%%%%%%%%%%%%%%%%%%% Problem 4
\section*{Problem 4}
\textbf{b} is the correct answer.

\noindent First of all, we know this is true because of the VC dimension + 1
theory. Secondly, for 5 points in $\mathbb{R}^3$, we get one of the following
situations. One situation is if all 5 points are on the same plane. Then
we obviously cannot shatter this, because we cannot even shatter 4 points in the
$\mathbb{R}^2$ case. Another situation is if 4 points are on the same plane
and one is on a different plane. We obviously cannot shatter this either,
since we could not shatter 4 points in $\mathbb{R}^2$.
In our last situation, we can form a plane with 3 points that separates the
2 other points. Then, we have that the dichotomy where the 3 points in that
plane are +1, the 1 point on one side of that plane is -1, and the 1 point on
the other side of the plane is also -1 cannot be achieved with the Perceptron Model.
%%%%%%%%%%%%%%%%%%%%%%%%%%%%%% Problem 5
\section*{Problem 5}
\textbf{b} is the correct answer.

\noindent We have that if there is no break point, $m_{\mathcal{H}}(N) = 2^N$,
and that if there is any break point, $m_{\mathcal{H}}(N)$ is polynomial in $N$.
So, we know that (i) and (ii) are actual growth functions, since they are
the growth functions for positive rays and positive intervals and are of the
form $\sum_{i=0}^{k-1}\binom{N}{i}$ (polynomial). And we have that (v) is just
$2^N$. Then we have that (iii) and (iv) are not polynomial in $N$ and not $2^N$.
So we have our answer.
%%%%%%%%%%%%%%%%%%%%%%%%%%%%%% Problem 6
\section*{Problem 6}
\textbf{c} is the correct answer.

\noindent We have that with the "2-invervals" learning model, we can have at
most 2 distinct sets of positive points. This is enough to shatter 3 and 4 points,
as it is impossible to have 3 distinct sets of positive points with just
3 or 4 points. But with 5 points, it is possible to have this: +1, -1, +1, -1,
+1, and we cannot achieve this dichotomy with the "2-intervals" learning
model. So the answer is 5.
%%%%%%%%%%%%%%%%%%%%%%%%%%%%%% Problem 7
\section*{Problem 7}
\textbf{c} is the correct answer.

\noindent We get to this answer the following way. There are $\binom{N+1}{4}$
ways to place 2 distinct intervals (choosing 4 bounds in total),
and $\binom{N+1}{2}$ ways to place 1 distinct inverval (choosing 2 bounds in
total). Then there is just $1$ dichotomy in which all the intervals are placed
together so that all the points are negative.
%%%%%%%%%%%%%%%%%%%%%%%%%%%%%% Problem 8
\section*{Problem 8}
\textbf{d} is the correct answer.

\noindent \textbf{d} is the only option that is consistent with the "1-interval"
and "2-interval" learning models - for the "1-interval" learning model we had
a break point of 3, and for the "2-interval" learning model we had a break point
of 5. This also makes sense in general, as given $M$ intervals to play with,
we can have at most $M$ distinct sets of positive points. So we are going to break
once we reach $2M + 1$ points, because at this point we can have $M+1$ distinct
sets of positive points as a possible dichotomy.
%%%%%%%%%%%%%%%%%%%%%%%%%%%%%% Problem 9
\section*{Problem 9}
\textbf{c} is the correct answer.

\noindent Basically solved this problem by doing the brute force way, trying
each answer one by one. 1 point can clearly be shattered. 3 points can also
clearly be shattered because a triangle is more powerful than a line. If we
arrange all the points in a circle, it is clear that 5 points can be shattered
as well. Trying to shatter 7 points (arranged in a circle, which should give
us the most possible dichotomies since a triangle is a convex set) fails.
So we have that our answer is 5.

%%%%%%%%%%%%%%%%%%%%%%%%%%%%%% Problem 10
\section*{Problem 10}
\textbf{e} is the correct answer.

\noindent Here, the maximum number of dichotomies we can get on $N$ points
is when we put the $N$ points on a line. When we do this, the concentric circles
act as intervals, and we can then see that this problem breaks down to
the "2-intervals" model. Thus our growth function for this is
the same as our growth function for the single positive interval problem,
$\binom{N+1}{4} + \binom{N+1}{2} + 1$. But this answer is not here, so it
is none of the above.
\end{document}
