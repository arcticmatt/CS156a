%%%%%%%%%%%%%%%%%%%%%%%%%%%%%% Preamble
\documentclass{article}
\usepackage{amsmath,amssymb,amsthm,fullpage}
\usepackage[a4paper,bindingoffset=0in,left=1in,right=1in,top=1in,
bottom=1in,footskip=0in]{geometry}
\newtheorem*{prop}{Proposition}
%\newcounter{Examplecount}
%\setcounter{Examplecount}{0}
\newenvironment{discussion}{\noindent Discussion.}{}
\setlength{\headheight}{12pt}
\setlength{\headsep}{10pt}
\usepackage{fancyhdr}

\pagestyle{fancy}
\fancyhf{}
\lhead{CS156a Pset 3}
\rhead{Matt Lim}
\begin{document}
%%%%%%%%%%%%%%%%%%%%%%%%%%%%%% Problem 1
\section*{Problem 1}
\textbf{b} is the correct answer.

\noindent We want the probability bound $2Me^{2\epsilon^2N}$ to be at most $.03$.
So, we can solve for
\[ 2Me^{-2\epsilon^2N} = .03 \]
\[ 2e^{-2(.05)^2N} = .03 \]
\[ e^{-2(.05)^2N} = .015 \]
\[ N = 839.94 \]
So for any $N$ less than this value, our probability bound will be greater
than $.03$. So the answer is $1000$. We can check this by plugging in
$500$ and $1000$ into $2Me^{2\epsilon^2N}$. At $500$, we get $.164$, which is
too big. At $1000$, we get $.013$, which is below the desired bound. Thus
$1000$ is the correct answer.
%%%%%%%%%%%%%%%%%%%%%%%%%%%%%% Problem 2
\section*{Problem 2}
\textbf{c} is the correct answer.

\noindent We want the probability bound $2Me^{2\epsilon^2N}$ to be at most $.03$.
So, we can solve for
\[ 2Me^{-2\epsilon^2N} = .03 \]
\[ 20e^{-2(.05)^2N} = .03 \]
\[ e^{-2(.05)^2N} = .0015 \]
\[ N = 1300.46 \]
So for any $N$ less than this value, our probability bound will be greater
than $.03$. So the answer is $1500$. We can check this by plugging in
$1000$ and $1500$ into $2Me^{2\epsilon^2N}$. At $1000$, we get $.135$, which is
too big. At $1500$, we get $.011$, which is below the desired bound. Thus
$1500$ is the correct answer.
%%%%%%%%%%%%%%%%%%%%%%%%%%%%%% Problem 3
\section*{Problem 3}
\textbf{d} is the correct answer.

\noindent We want the probability bound $2Me^{2\epsilon^2N}$ to be at most $.03$.
So, we can solve for
\[ 2Me^{-2\epsilon^2N} = .03 \]
\[ 200e^{-2(.05)^2N} = .03 \]
\[ e^{-2(.05)^2N} = .00015 \]
\[ N = 1760.98 \]
So for any $N$ less than this value, our probability bound will be greater
than $.03$. So the answer is $2000$. We can check this by plugging in
$1500$ and $2000$ into $2Me^{2\epsilon^2N}$. At $1500$, we get $.111$, which is
too big. At $2000$, we get $.009$, which is below the desired bound. Thus
$2000$ is the correct answer.
%%%%%%%%%%%%%%%%%%%%%%%%%%%%%% Problem 4
\section*{Problem 4}
%%%%%%%%%%%%%%%%%%%%%%%%%%%%%% Problem 5
\section*{Problem 5}
%%%%%%%%%%%%%%%%%%%%%%%%%%%%%% Problem 6
\section*{Problem 6}
%%%%%%%%%%%%%%%%%%%%%%%%%%%%%% Problem 7
\section*{Problem 7}
%%%%%%%%%%%%%%%%%%%%%%%%%%%%%% Problem 8
\section*{Problem 8}
%%%%%%%%%%%%%%%%%%%%%%%%%%%%%% Problem 9
\section*{Problem 9}
%%%%%%%%%%%%%%%%%%%%%%%%%%%%%% Problem 10
\section*{Problem 10}
\end{document}
